\capitulo{1}{Introducción}

El término Inteligencia Artificial (IA) fue acuñado por primera vez en la Conferencia de Dartmouth hace ahora 65 años, esto es, en 1956 \cite{crevier95}. Sin embargo, ha sido en los últimos años cuando su presencia e importancia en la sociedad han crecido de manera exponencial.

Uno de los campos históricos dentro de la AI, es el Procesamiento del Lenguaje Natural (NLP, por sus siglas en inglés), cuya significación se hizo patente con la aparición del célebre Test de Turing \cite{turing50}, en el cual un interrogador debe discernir entre un humano y una máquina conversando con ambos por escrito a través de una terminal.

Hasta los años 80, la mayor parte de los sistemas de NLP estaban basados en complejas reglas escritas manualmente \cite{mccorduck79}, las cuales conseguían generalmente modelos muy lentos, poco flexibles y con baja precisión. A partir de esta década, como fruto de los avances en Aprendizaje Automático (\emph{Machine Learning}), fueron apareciendo modelos estadísticos, consiguiendo notables avances en campos como el de la traducción automática.

En la última década, el desarrollo ha sido aún mayor debido a factores como el aumento masivo de datos de entrenamiento (principalmente provenientes de la \emph{web}), avances en la capacidad de computación (\emph{Graphic Processing Units} o GPUs) y el progreso dentro del área de la Algoritmia \cite{rahmfeld19}.

No obstante, ha sido desde la aparición del concepto de ``atención'' en 2015 \cite{luong15, bahdanau16} cuando el campo del NLP ha comenzado a conseguir resultados cuanto menos sorprendentes \cite{macaulay20, wiggers21}.

TODO: por qué JIZT (acercar estos modelos al público general).