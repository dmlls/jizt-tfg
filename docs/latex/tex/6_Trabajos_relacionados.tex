\capitulo{6}{Trabajos relacionados}

\section{Artículos académicos}

Las publicaciones más relevantes para nuestro proyecto están relacionadas con los modelos de los que hacemos uso para la generación de resúmenes: T5 y Truecase.

\subsubsection{\emph{Exploring the Limits of Transfer Learning with a Unified Text-to-Text Transformer}}

En este artículo, Colin Raffel \emph{et al.} \cite{raffel19}, de Google, estudian las ventajas de la técnica del aprendizaje por transferencia (\emph{transfer learning}) al campo del Procesamiento del Lenguaje Natural (NLP). Tradicionalmente, cada nuevo modelo se entrenaba desde cero. Esto ha cambiado con la inclusión del aprendizaje por transferencia; actualmente, la tendencia es emplear modelos pre-entrenados como punto de partida para la construcción de nuevos modelos.

Las tres principales ventajas del empleo del aprendizaje por transferencia son \cite{sarkar18}:

\vspace*{-\baselineskip}
\begin{itemize}
	\item [\textbullet] Mejora del rendimiento de partida. El hecho de partir desde un modelo pre-entrenado, frente a un modelo ignorante o \emph{ignorant learner}, proporciona un rendimiento base desde el primer momento.

	\item [\textbullet] Disminución del tiempo de desarrollo del modelo, derivado del punto anterior.
	
	\item [\textbullet] Mejora del rendimiento final. Esta mejora ha sido estudiada tanto en el caso del NLP \cite{kumar21}, como de otros ámbitos, como la visión artificial \cite{ali21}, o el campo de la medicina \cite{liu21}.
\end{itemize}

La principal novedad de este artículo se encuentra en su propuesta de tratar todos los problemas de procesamiento de texto como problemas texto a texto (\emph{text-to-text}), es decir, tomar un texto como entrada, y producir un nuevo texto como salida. Esto permite crear un modelo general, al que han bautizado como T5, capaz de llevar a cabo diversas tareas de NLP, como muestra el siguiente diagrama:

\bigskip

\imagen{t5-paper}{El \emph{framework} texto a texto permite emplear el mismo modelo, con los mismos hiperparámetros, función de pérdida, etc., para aplicarlo a diversas tareas de NLP \cite{raffel19}.}

En cualquier caso, se puede realizar un ajuste fino del modelo para una de las tareas, a fin de mejorar su rendimiento en dicha tarea específica.

Las posibilidades que este modelo nos ofrece son muy interesantes, dado que en un futuro, nuestro proyecto podría incluir otras tareas de Procesamiento de Lenguaje Natural, haciendo uso de un solo modelo.

\bigskip
\subsubsection{\emph{tRuEcasIng}}

En este artículo, fruto de la colaboración entre la universidad Carnegie Mellon (Pensilvania, EE. UU.) e IBM, los autores Lucian Vlad Lita \emph{et al.}, exploran los problemas del \emph{truecasing}, es decir, el proceso de recomponer las mayúsculas de un texto, y proponen un \emph{truecaser} estadístico que alcanza una precisión del 98\% en artículos de noticias \cite{lita03}.


\section{Proyectos similares}

$x$

\begin{table}[h]\label{tabla:comparativa}
	\centering
	\begin{tabular}{@{}lcccc@{}}
		\toprule
		\multirow{2}{*}{Caraterísticas} & \multirow{2}{*}{\textbf{JIZT}} & \scriptsize{Bert Extracive} & \multirow{2}{*}{\small{ExplainToMe}} & \small{EssayToolBox} \\
		& & \scriptsize{Summarizer} & & \small{Summarizer}
		\tabularnewline
		\midrule
		Plataformas & Web & Web & Web & Web\tabularnewline
		\emph{Open-source} & PHP & PHP & Java &
		Java\tabularnewline
		Tipo de resumen & \cellcolor{green!25} {$\checkmark$} & \cellcolor{green!25} {$\checkmark$} & \cellcolor{green!25} {$\checkmark$} & \cellcolor{green!25} {$\checkmark$}\tabularnewline
		API REST & \cellcolor{green!25} {$\checkmark$} & \cellcolor{green!25} {$\checkmark$} & \cellcolor{green!25} {$\checkmark$} & \cellcolor{green!25} {$\checkmark$}\tabularnewline
		Arquitectura & \cellcolor{green!25} {$\checkmark$} & \cellcolor{green!25} {$\checkmark$} & \cellcolor{green!25} {$\checkmark$} & \cellcolor{green!25} {$\checkmark$}\tabularnewline
		Tiempo de generación & \cellcolor{green!25} {$\checkmark$} & \cellcolor{red!25} {$\times$} & \cellcolor{red!25} {$\times$} & \cellcolor{red!25} {$\times$}\tabularnewline
		Múltiples formatos de localización & \cellcolor{green!25} {$\checkmark$} & \cellcolor{red!25} {$\times$} & \cellcolor{red!25} {$\times$} &\cellcolor{red!25} {$\times$}\tabularnewline
		Cobertura temporal & \cellcolor{red!25} {$\times$} & \cellcolor{green!25} {$\checkmark$} & \cellcolor{red!25} {$\times$} & \cellcolor{green!25} {$\checkmark$}\tabularnewline
		Protocolo \emph{OAI-PMH} & \cellcolor{green!25} {$\checkmark$} & \cellcolor{green!25} {$\checkmark$} & \cellcolor{green!25} {$\checkmark$} & \cellcolor{red!25} {$\times$}\tabularnewline
		Soporte para \emph{ARIADNEplus} & \cellcolor{green!25} {$\checkmark$} & \cellcolor{red!25} {$\times$} & \cellcolor{red!25} {$\times$} &
		\cellcolor{red!25} {$\times$}\tabularnewline
		Transformación de metadatos & \cellcolor{green!25} {$\checkmark$} & \cellcolor{green!25} {$\checkmark$} & \cellcolor{red!25} {$\times$} &
		\cellcolor{red!25} {$\times$}\tabularnewline
		Sistema de usuarios & \cellcolor{green!25} {$\checkmark$} & \cellcolor{green!25} {$\checkmark$} & \cellcolor{green!25} {$\checkmark$} & \cellcolor{green!25} {$\checkmark$}\tabularnewline
		Almacenamiento de ficheros & \cellcolor{green!25} {$\checkmark$} & \cellcolor{green!25} {$\checkmark$} & \cellcolor{green!25} {$\checkmark$} & \cellcolor{red!25} {$\times$}\tabularnewline
		Asistencia técnica gratuita & \cellcolor{green!25} {$\checkmark$} & \cellcolor{red!25} {$\times$} & \cellcolor{red!25} {$\times$} & \cellcolor{red!25} {$\times$}\tabularnewline
		Interfaz pública & \cellcolor{green!25} {$\checkmark$} & \cellcolor{green!25} {$\checkmark$} & \cellcolor{green!25} {$\checkmark$} & \cellcolor{green!25} {$\checkmark$}\tabularnewline
		Interfaz intuitiva & \cellcolor{green!25} {$\checkmark$} & \cellcolor{red!25} {$\times$} & \cellcolor{green!25} {$\checkmark$} & \cellcolor{red!25} {$\times$}\tabularnewline
		Sistema de \emph{plugins} & \cellcolor{green!25} {$\checkmark$} & \cellcolor{red!25} {$\times$} & \cellcolor{green!25} {$\checkmark$} (*) & \cellcolor{red!25} {$\times$}\tabularnewline
		Sistema de plantillas & \cellcolor{green!25} {$\checkmark$} & \cellcolor{red!25} {$\times$} & \cellcolor{red!25} {$\times$} & \cellcolor{red!25} {$\times$}\tabularnewline
		Comunidad de usuarios activa & \cellcolor{green!25} {$\checkmark$} & \cellcolor{red!25} {$\times$} & \cellcolor{green!25} {$\checkmark$} & \cellcolor{red!25} {$\times$}\tabularnewline
		Manuales de documentación detallados & \cellcolor{green!25} {$\checkmark$} & \cellcolor{red!25} {$\times$} & \cellcolor{red!25} {$\times$} & \cellcolor{red!25} {$\times$}\tabularnewline
		Última actualización & 2020 & 2018 & 2020 & 2015\tabularnewline
		\bottomrule
	\end{tabular}
	\caption{Comparativa de las características de las aplicaciones propuestas por cada socio.}
\end{table}
