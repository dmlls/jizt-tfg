\capitulo{4}{Técnicas y herramientas} \label{chapter:tecnicas}

\section{Flask y Flask-RESTful}

Flask es uno de los \emph{frameworks} más populares para la creación de aplicaciones \emph{web} en Python\cite{flask}. Está concebido para ser lo más simple posible. En nuestro caso, la hemos empleado para implementar la lógica de la API REST. Además, hemos utilizado una conocida extensión de Flask, Flask-RESTful \cite{flaskRestful}, que facilita aún más dicha implementación.


\section{Docker}

Se trata de una serie de servicios como plataforma (PaaS), que proporcionan virtualización a nivel de sistema operativo, permitiendo ejecutar \emph{software} en paquetes llamados \emph{contenedores} \cite{docker}.

A diferencia de las máquinas virtuales, en las cuales el sistema operativo subyacente se comparte a través del hipervisor, cada contenedor Docker ejecuta su propio sistema operativo.

Docker nos va a permitir encapsular cada servicio en un contenedor, posibilitando la implementación de la arquitectura de microservicios.



\section{Kubernetes}


